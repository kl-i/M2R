In the following section, let $p \in \Z$, $0 < p$ be a prime. 

\subsection{\texorpdfstring{$\Z_p$ and $\Q_p$}{TEXT}}

\begin{dfn}[Projective System]\link{dfn:proj_sys}
  
  Let $\N$ be the naturals viewed as a category with the usual ordering. 
  Let $\CC$ be a category. 
  Then a \emph{projective system in $\CC$} is a contravariant functor 
  from $\N$ to $\CC$. 
  For a projective system $F$, 
  we will denote the image of the morphism $k \leq l$ with $\fall{k}{l}$. 

  Equivalently, 
  a projective system in $\CC$ is 
  a collection of objects $(F_n)_{n\in\N}$ in $\CC$ together with
  a collection of maps $(\fall{n}{n+1} : \map{F_{n+1}}{F_n}{}{})_{n\in \N}$
  such that for all $n \in \N$, $\fall{n}{n+1}\fall{n+1}{n+2} = \fall{n}{n+2}$.
\end{dfn}

\begin{dfn}[Surjective System]\link{dfn:surj_sys}
  
  Let $R$ be a commutative ring and $F$ a projective system of $R$-modules. 
  Then $F$ is called \emph{surjective} when 
  for all $n \in \N$, $\fall{n}{n+1}$ is surjective. 
\end{dfn}

\begin{dfn}[Inverse Limit of a Projective System]\link{dfn:inv_lim}
  
  Let $\CC$ be a category and $F : \map{\N\op}{\CC}{}{}$ be a projective system.
  Then an \emph{inverse limit of $F$} is just 
  a limit of $F$ as an $\N\op$-diagram.

  More explicitly, an inverse limit of $F$ is 
  an object $A$ of $\CC$ together with a collection of maps 
  $(\al_n : A \to F_n)_{n\in\N}$ such that 
  for all $n \in \N$, $\fall{n}{n+1}\al_{n+1} = \al_n$. 
\end{dfn}

\begin{lem}[Left Surjective implies Right Exactness of Inverse Limit]
  \link{prop:inv_lim:left_surj_imp_right_exact}
  
  Let $R$ be a ring and \[
    0 \to A \to B \to C \to 0
  \]
  be a short exact sequence of projective systems of $R$-modules,
  i.e. for all $n \in \N$, 
  \[
    0 \to A_n \to B_n \to C_n \to 0
  \]
  is a short exact sequence. 
  Then \[
    0 \to \LIM A \to \LIM B \to \LIM C \to 0
  \]
  is exact at $\LIM A$ and $\LIM B$. 
  Furthermore, if $A$ is surjective, 
  then we also have exactness at $\LIM C$. 

\end{lem}

\begin{dfn}[$p$-adic Integers]\link{dfn:padic}
  
  Define the following projective system of rings, 
  $\Z/p^\star\Z$ by : 
  \begin{enumerate}
    \item $n \in \obj{\N\op} \mapsto \Z/p^n\Z$
    \item For $n \in \N\op$, $\fall{n}{n+1} : \map{\Z/p^{n+1}\Z}{\Z/p^n\Z}{}{}$
    is the natural projection (from the universal property of $\Z/p^{n+1}\Z$). 
  \end{enumerate}
  Then the \emph{$p$-adic integers} $\Z_p$ is defined as 
  the inverse limit of $\Z/p^\star\Z$. 
  For $n \in \N$, $\ep_n : \map{\Z_p}{\Z/p^n\Z}{}{}$ will denote the 
  projection that comes with the definition of $\Z_p$ as a limit. 

  We have an explicit construction of $\Z_p$ as 
  the subset of $x \in \dprod{n\in\N}{} \Z/p^n\Z$ such that 
  for all $n \in \N$, $\fall{n}{n+1}\ep_{n+1}(x) = \ep_n(x)$,
  where $\ep_n : \dprod{n\in\N}{} \Z/p^n\Z \to \zmo{n}$ is the projection into
  the $n$-th component. 
\end{dfn}

\begin{rmk}[Meaning of $p$-adic integers]\link{rmk:padic}
  
  One should think of $p$-adic integers along the following analogy with 
  complex analysis : 
  \begin{enumerate}
    \item $\Z$ is the ring of holomorphic functions on a space,
    the space being the set of primes of $\Z$. 
    \item A prime $p$ is a point. 
    \item Taking an integer $f$ to $\Z/p\Z$ is evaluation of 
    the function $f$ at the point $p$. 
    \item Sending an integer $f$ to $\Z/p^n\Z$ is 
    the taylor expansion of $f$ at $p$
    up to terms of order $n$.
    You can write $f$ in $\Z/p^n\Z$ as a polynomial in $1,p,\dots,p^{n-1}$
    with coefficients in $\set{0,\dots,p-1}$. 
    \item Elements of $\Z_p$ are precisely 
    coherent collections of taylor expansions of higher and higher order,
    i.e. power series in $p$. 
    This is formalized \linkto{prop:padic:top_prop}{later}.
  \end{enumerate}
\end{rmk}

\begin{prop}[$\Z$ injects into $\Z_p$]\link{prop:padic:inject}
  
  The canonical ring morphism $\map{\Z}{\Z_p}{}{}$ has kernel 
  $\bigcap_{n \in \N} p^n\Z = 0$.
\end{prop}
\begin{proof}
  
  We have a short exact sequence of projective systems of $\Z$-modules, 
  \begin{figure}[H]
    \centering
    \begin{tikzcd}
      0 \arrow[r] & p^\star\Z \arrow[r] 
      & \underline{\Z} \arrow[r] & \Z/p^\star\Z \arrow[r] & 0
    \end{tikzcd}
  \end{figure}
  where the middle projective system is a constant at $\Z$. 
  Since \linkto{prop:inv_lim:left_surj_imp_right_exact}{
  taking inverse limits is left exact}, we obtain the following 
  exact sequence of $\Z$ modules : 
  \begin{figure}[H]
    \centering
    \begin{tikzcd}
      0 \arrow[r] & \LIM p^\star\Z \arrow[r] 
      & \Z \arrow[r] & \LIM \Z/p^\star\Z
    \end{tikzcd}
  \end{figure} 
  Since the forgetful functor from the category of rings to 
  $\Z$-modules is a right adjoint functor, it preserves limits.
  In particular, 
  the inverse limit of $\Z/p^\star\Z$ in the category of $\Z$-modules is 
  still $\Z_p$.
  It is easy to check that the inverse limit of $p^\star\Z$ is the intersection.
\end{proof}

\begin{prop}[Truncation]\link{prop:padic:truncation}
  
  Let $n \in \N$. 
  Then we have the following short exact sequence of $\Z$-modules : 
  \begin{figure}[H]
    \centering
    \begin{tikzcd}
      0 \arrow[r] & 
      \Z_p \arrow[r,"p^n"] & 
      \Z_p \arrow[r,"\ep_n"] & 
      \Z/p^n\Z \arrow[r] & 
      0
    \end{tikzcd}
  \end{figure}
\end{prop}
\begin{proof}(Generalized from nLab)

  %Consider the following short exact sequence 
  %of projective systems in $\MOD[\Z]$ : 
  %\begin{figure}[H]
  %  \centering
  %  \begin{tikzcd}
  %    & \vdots \arrow[d] & \vdots \arrow[d] & \vdots \arrow[d] & \\
  %    0 \arrow[r] & \Z/p^2\Z \arrow[d]\arrow[r,"p^n"] 
  %      & \Z/p^{n+2}\Z \arrow[d]\arrow[r] & \Z/p^n\Z \arrow[d]\arrow[r] & 0 \\
  %    0 \arrow[r] & \Z/p\Z \arrow[d]\arrow[r,"p^n"] 
  %      & \Z/p^{n+1}\Z \arrow[d]\arrow[r] & \Z/p^n\Z \arrow[d]\arrow[r] & 0 \\
  %    0 \arrow[r] & 0 \arrow[d]\arrow[r] 
  %      & \Z/p^n\Z \arrow[d]\arrow[r] & \Z/p^n\Z \arrow[d]\arrow[r] & 0 \\
  %    0 \arrow[r] & 0 \arrow[d]\arrow[r] & \Z/p^{n-1}\Z \arrow[d]\arrow[r] 
  %      & \Z/p^{n-1}\Z \arrow[d]\arrow[r]& 0 \\
  %    & \vdots \arrow[d] & \vdots \arrow[d] & \vdots \arrow[d] & \\
  %    0 \arrow[r] & 0 \arrow[d]\arrow[r] & \Z/p\Z \arrow[d]\arrow[r]
  %      & \Z/p\Z \arrow[d]\arrow[r] & 0 \\
  %    0 \arrow[r] & 0 \arrow[r] & 0 \arrow[r] & 0 \arrow[r] & 0
  %  \end{tikzcd}
  %\end{figure}
  Consider the following short exact sequence of projective systems of 
  $\Z$-modules : 
  \begin{cd}[sep = small]
    0 \arrow[d] & 
    0 \arrow[l] \arrow[d] &
    \cdots \arrow[l] \arrow[d] & 
    0 \arrow[l] \arrow[d] & 
    \zmo{} \arrow[l] \arrow[d,"p^n"] & 
    \zmo{2} \arrow[l] \arrow[d,"p^n"] & 
    \cdots \arrow[l] \\
    0 \arrow[d] & 
    \zmo{} \arrow[l] \arrow[d] &
    \cdots \arrow[l] \arrow[d] & 
    \zmo{n} \arrow[l] \arrow[d] & 
    \zmo{n+1} \arrow[l] \arrow[d] & 
    \zmo{n+2} \arrow[l] \arrow[d] & 
    \cdots \arrow[l] \\
    0 & 
    \zmo{} \arrow[l] &
    \cdots \arrow[l] & 
    \zmo{n} \arrow[l] & 
    \zmo{n} \arrow[l] & 
    \zmo{n} \arrow[l] & 
    \cdots \arrow[l] \\
  \end{cd}
  \linkto{prop:inv_lim:left_surj_imp_right_exact}
  {Since the left system is surjective}, 
  by taking inverse limits we obtain 
  the desired short exact sequence of $\Z$-modules : 
  \begin{figure}[H]
    \centering
    \begin{tikzcd}
      0 \arrow[r] & 
      \Z_p \arrow[r,"p^n"] & 
      \Z_p \arrow[r,"\ep_n"] & 
      \Z/p^n\Z \arrow[r] & 
      0
    \end{tikzcd}
  \end{figure}
\end{proof}

\begin{rmk}[Meaning of Truncation]\link{rmk:truncation}
  
  $\ep_n$ is precisely truncating a power series at 
  terms of order $n$ and higher.
  Then the theorem says the power series that are zero up to terms order $n$
  are precisely the ones consisting of terms of order $n$ and higher. 
\end{rmk}

\begin{prop}[$\Z_p$ Local Ring]\link{prop:padic:local}
  
  $\Z_p$ is a local ring with maximal ideal $p\Z_p$. 
\end{prop}
\begin{proof}(via geometric series)

  We show $\Z_p$ is local directly. 
  Since $p\Z_p = \ker \ep_1$ which is a maximal ideal in $\Z_p$, 
  it suffices that $p\Z_p$ is a subset of the Jacobson radical of $\Z_p$,
  equivalently $1 - p\Z_p \subs \Z_p^\times$. 

  Let $x \in p\Z_p$. 
  All we need to do is justify $1/(1 - x) = \dsum{k = 0}{\infty} x^k$
  is an element in $\Z_p$. 
  For $k \in \N$, define $y_k := \dsum{0 \leq l < k}{} \ep_k(x^l) \in \Z/p^k\Z$
  and let $y$ be the unique element in $\dprod{n \in \N}{}\Z/p^n\Z$ such that 
  for all $k \in \N$, $\ep_k(y) = y_k$. 
  Then $x \in p\Z_p$ implies $x^k \in p^k\Z_p = \ker \ep_k$, 
  which shows that $y \in \Z_p$ and is the desired inverse of $1 - x$. 
\end{proof}

\begin{rmk}[Why $\Z_p$ is a Local Ring]\link{rmk:padic:local}
 
  This is the analogue of the fact that a power series is invertible 
  if and only if its constant coefficient is invertible. 
\end{rmk}

We will now give $\Z_p$ a norm that makes precise 
the intuition that higher order terms tend to zero. 

\begin{dfn}[Naturals with Infinity]\link{dfn:nat_infty}
  
  Let $\N^\infty := \N \sqcup \set{\infty}$.
  Define $\leq$ on $\N^\infty$ as follows : 
  \begin{itemize}
    \item[--] For all $n, m \in \N^\infty\minus\set{\infty}$, 
    $n \leq m$ is as usual. 
    \item[--] For all $n \in \N^\infty$, $n \leq \infty$. 
  \end{itemize}

  Define $+$ on $\N$ as follows : 
  \begin{itemize}
    \item[--] For $n, m \in \N^\infty\minus\set{\infty}$, $n + m$ is as usual. 
    \item[--] For $n \in \N^\infty$, $n + \infty = \infty$. 
  \end{itemize}
\end{dfn}

\begin{dfn}[$p$-adic Valuation, Norm]\link{dfn:padic:val}
  
  The \emph{$p$-adic valuation} is defined as the following : 
  \begin{align*}
    v_p : \map{\Z_p}{\N^\infty}{}{}, 
    x \mapsto \sup\set{n \in \N^\infty \st \ep_n(x) = 0}
  \end{align*}

  From this, we define the \emph{$p$-adic norm}, 
  \begin{align*}
    \abs{\hspace{5pt}}_p : \map{\Z_p}{[0,\infty)\subs \R}{}{}, 
    x \mapsto \begin{cases}
      p^{-v_p(x)} &,\, x \neq 0 \\
      0 &,\, x = 0
    \end{cases}
  \end{align*}
\end{dfn}

\begin{rmk}[Meaning of $p$-adic Norm]\link{rmk:padic:norm}
  
  Under the interpretation of $p$-adic integers as power series, 
  $v_p(x)$ is the lowest power of $p$ with non-zero coefficient. 
\end{rmk}

\begin{prop}[Unique Decomposition in $\Z_p$]\link{prop:padic:decomp}
  
  Let $x \in \Z_p$, $x \neq 0$. 
  Then \begin{enumerate}
    \item $v_p(x) \neq \infty$. 
    \item Since by definition, $\ep_{v_p(x)}(x) = 0$ and 
    multiplying by $p^{v_p(x)}$ is injective, 
    there exists a unique $u(x) \in \Z_p$ such that $x = p^{v_p(x)}u(x)$.
    Then $u(x) \in \Z_p^\times$
    \item For all $n \in \N$ and $u \in \Z_p^\times$, 
    $x = p^n u$ implies $n = v_p(x)$ and $u = u(x)$
  \end{enumerate}
\end{prop}
\begin{proof}~

  $(1)$ For $n \in \N$, $\ep_n(x) = 0$ implies for all $k \leq n$, 
  $\ep_k(x) = 0$. 
  Since $x \neq 0$, this implies the set of $n$ such that 
  $\ep_n(x) = 0$ is bounded above by a natural $N \in \N$. 
  Hence $v_p(x) \leq N < \infty$. 
  
  $(2)$ Since \linkto{prop:padic:local}{$\Z_p$ is a local ring} with 
  maximal ideal $p\Z_p$, it suffices to show that 
  $u(x) \notin p\Z_p = \ker\ep_1$. 
  Well, if $u(x) \in p\Z_p$, then $x \in p^{v_p(x)}+1\Z_p$, 
  which implies $\ep_{v_p(x) + 1}(x) = 0$, 
  contradicting the maximality of $v_p(x)$. 

  $(3)$ Let $n \in \N$, $u \in \Z_p^\times$ such that $x = p^n u$. 
  Already, $x \in p^n\Z_p$ implies $n \leq v_p(x)$ by 
  \linkto{dfn:padic:val}{definition of $v_p(x)$}. 
  Then $u \in p^{v_p(x) - n}\Z_p$ and $u \in \Z_p^\times$
  implies $v_p(x) = n$. 
  Then $u = u(x)$ since multiplying by $p^{v_p(x)}$ is 
  \linkto{prop:padic:truncation}{injective}. 
\end{proof}

\begin{prop}[$(\Z_p,\abs{\hspace{5pt}}_p)$ Normed Ring]
  \link{prop:padic:normed_ring}
  
  The following are true : 
  \begin{enumerate}
    \item (Positive Definite) 
    For all $x \in \Z_p$, $\abs{x}_p = 0$ if and only if $x = 0$. 
    \item (Ultrametric Property) 
    For all $x, y \in \Z_p$, 
    $\abs{x + y}_p \leq \max(\abs{x}_p,\abs{y}_p)$. 
    \item (Multiplcative)
    For $x, y \in \Z_p$, 
    $\abs{x y}_p = \abs{x}_p \abs{y}_p$. 
    \item (Normalized)
    $\abs{1}_p = 1$
  \end{enumerate}
  Hence $\Z_p$ is a topological ring with the topology from $\abs{\hspace{5pt}}_p$. 
\end{prop}
\begin{proof}~
  
  $(1)$ Clear. 

  $(2)$ It suffices to show $\min(v_p(x),v_p(y)) \leq v_p(x + y)$. 
  Let $n = \min(v_p(x),v_p(y))$. 
  Then $\ep_n(x + y) = \ep_n(x) + \ep_n(y) = 0$.
  So $n \leq v_p(x + y)$ by its maximality. 

  $(3)$ It suffices to show $v_p(x y) = v_p(x) + v_p(y)$. 
  This follows from the result on 
  \linkto{prop:padic:decomp}{unique decomposition}.

  $(4)$ $v_p(1) = 0$ since $1$ is a unit. 
\end{proof}

\begin{prop}[$\Z_p$ Integral Domain]\link{prop:padic:ID}
  
  For all $x, y \in \Z_p$, $x y = 0$ implies $x = 0$ or $y = 0$. 
\end{prop}
\begin{proof}
  Follows from the norm being \linkto{prop:padic:normed_ring}{multipicative}
  and $\R$ being an integral domain. 
\end{proof}
\begin{proof}(Without using the norm)

  Let $x, y \in \Z_p$, $x \neq 0 \neq y$. 
  Then $x y = p^{v_p(x) + v_p(y)} u(x) u(y)$ 
  where $u(x), u(y) \in \Z_p^\times$ from 
  \linkto{prop:padic:decomp}{unique decomposition}. 
  Then $x y = 0$ yields $0 = p^{v_p(x) + v_p(y)}$, 
  which implies $\Z$ does not \linkto{prop:padic:inject}{inject} into $\Z_p$,
  a contradiction. 
\end{proof}

\begin{prop}[Ultrametric Property]\link{prop:padic:ultrametric}
  
  Let $(X,d)$ be a metric space with 
  $d$ satisfying the \emph{ultrametric property} : 
  for all $x, y, z \in X$, $d(x,z) \leq \max(d(x,y),d(y,z))$.
  Then for all sequences $a : \N \to X$, 
  $a_n$ is cauchy if and only if $\lim_{n\to\infty} d(a_n,a_{n+1}) = 0$.
\end{prop}
\begin{proof}
  Elementary. 
\end{proof}

\begin{prop}[Topological Properties of $\Z_p$]\link{prop:padic:top_prop}

  The following are true : 
  \begin{enumerate}
    \item (Topology) 
    Give $\Z_p$ the subspace topology in $\dprod{n \in \N}{}\Z/p^n\Z$ with 
    the product topology from each $\Z/p^n\Z$ being discrete. 
    Then for all $x \in \Z_p$, 
    the set of balls $\set{B_{p^{-n}(x)}}_{n \in \N}$ is 
    a prefilter that generates the neighbourhood filter of $x$
    (AKA a neighbourhood base). 
    That is to say, the topology from the norm is equal to 
    the topology from the construction of $\Z_p$. 
    \item (Completeness)
    $\Z_p$ is compact and
    hence a complete metric space under $\abs{\hspace{5pt}}_p$. 
    \item (Density of $\Z$ in $\Z_p$)
    For each $x \in \Z_p$, 
    there exists unique $a : \N \to \set{0,\dots,p-1}$ such that 
    $x = \dsum{k = 0}{\infty} a_k p^k$.
    Furthermore, for all $a : \N \to \set{0, \dots, p-1}$, 
    $\dsum{k = 0}{\infty} a_k p^k$ is convergent in $\Z_p$. 
  \end{enumerate}
\end{prop}
\begin{proof}~

  $(1)$ Let $x \in \Z_p$. 
  By the definition of product topology, 
  the neighbourhood filter of $x$ is generated by 
  the set of preimages of open neighbourhoods of $\ep_n(x)$, 
  where $n$ ranges over $\N$. 
  Since the $\Z/p^n\Z$ are all discrete, 
  the neighbourhood filter of $x$ is generated by the smaller set of 
  $\set{\ep_n\inv(\ep_n(x))}_{n \in \N} = \set{x + p^n\Z_p}_{n\in\N}
  = \set{B_{p^{-n+1}}(x)}_{n\in\N}$,
  hence the result.
  
  $(2)$ Define $C : \N \to 2^{\dprod{n\in\N}{}\zmo{n}}$
  by mapping $n \in \N$ to the set of elements $x$ such that 
  $\fall{n}{n+1}\ep_{n+1}(x) = \ep_n(x)$.
  Then $\Z_p = \bigcap_{n \in \N} C_n$. 
  Since $\dprod{n \in \N}{}\Z/p^n\Z$ is compact by Tychonoff's theorem
  and closed in compact implies compact, it suffices to show that
  each $C_n$ is closed. 
  We can describe $C_n$ explicitly as 
  \[
    C_n = \bigcup_{y \in \zmo{n}} 
    \bigcup_{z \in \brkt{\fall{n}{n+1}}\inv y} \ep_n\inv y \cap \ep_{n+1}\inv z
  \]
  Since every $\zmo{n}$ is discrete,
  this is a finite union of closed sets and hence is closed. 

  $(3)$ In the following, let $\pi_k : \Z \to \Z/p^k\Z$ be the natural map. 
  Let $x \in \Z_p$. 
  For $k \in \N$, let $x_k \in \Z$ be unique such that 
  $\pi_k(x_k) = \ep_k(x)$ and $0 \leq x_k < p^k$. 
  There exists a unique $a^{(k)} : \N \to \set{0,\dots,p-1}$ such that 
  $x_k = \dsum{l\in\N}{} a^{(k)}_l p^l$. 
  Since $\pi_k(x_{k+1} - a^{(k+1)}(k) p^k) = \pi_k(x_{k+1})
  = \fall{k}{k+1}\ep_{k+1}(x) = \ep_k(x) = x_k$
  and $0 \leq x_{k+1} - a^{(k+1)}(k) p^k < p^k$,
  we have $x_{k+1} = x_k + a^{(k+1)}(k) p^k$.
  Therefore $a : \N \to \set{0,\dots,p-1}, k \mapsto a^{(k)}(k)$. 
  The claim that $x = \dsum{k = 0}{\infty} a_k p^k$
  is equivalent to $x = \lim_{k \to \infty} x_k$. 
  Since the neighourhood filter of $x$ is generated by $B_{p^{-n}}(x)$,
  it suffices $x_k$ converges into each of these balls. 
  Let $n \in \N$. 
  Then for $k \geq n + 1$, 
  $\ep_{n+1}(x_k - x) = \fall{n+1}{k}\ep_k(x_k - x) = 0$.
  Therefore $n < v_p(x_k - x)$, i.e. $x_k \in B_{p^{-n}}(x)$.
  Hence, $x_k \to x$. 

  Let $b : \N \to \set{0,\dots,p-1}$ such that 
  $x = \dsum{k = 0}{\infty} b_kp^k$.
  Then $\pi_1(a_0) = \ep_1(x) = \pi_1(b_0)$.
  Since $0 \leq a_0,b_0 < p$, $a_0 = b_0$. 
  For $k \in \N$, 
  $\pi_{k+1}(a_{k}p^{k}) = \ep_{k+1}(x - \dsum{0 \leq l < k}{}a_lp^l)
  = \ep_{k+1}(x - \dsum{0 \leq l < k}{}b_lp^l) = \pi_{k+1}(b_{k}p^{k})$
  by induction.
  Since $0 \leq a_k, b_k < p$, $a_k p^k = b_k p^k$ and hence $a_k = b_k$.
  Therefore $a = b$. 

  A general power series in $p$ converges because 
  $\abs{a_k p^k}_p \leq \abs{p}_p^k = p^{-k} \to 0$,
  the \linkto{prop:padic:normed_ring}{ultrametric property} of the norm
  and completeness of $\Z_p$. 
\end{proof}

\begin{dfn}[$p$-adic Rationals]\link{dfn:padic:rat}
  
  $\Q_p$ is defined as the field of fractions of $\Z_p$. 
\end{dfn}

\begin{prop}[$\Q_p$ as Localizing $\Z_p$ at $p$]
  \link{prop:padic:rat:localization}
  
  As $\Z_p$ algebras,
  $\Q_p$ is canonically isomorphic to $(\Z_p)_p = \Z_p[X]/(pX-1)\Z_p[X]$,
  the localization of $\Z_p$ with respect to the element $p$. 
\end{prop}
\begin{proof}
  Since $p$ is invertible in $\Q_p$, 
  there is a canonical $\Z_p$-algebra morphism from $(\Z_p)_p$ to $\Q_p$.
  Since $\Z_p$ be an integral domain, 
  $\Z_p$ injects into $\Q_p$ and thus $(\Z_p)_p$ injects into $\Q_p$ as well.
  By \linkto{prop:padic:decomp}{unique decomposition},
  every element of $\Q_p$ is of the form $(p^nu)/(p^m v)$ where 
  $n, m \in \N$ and $u, v \in \Z_p^\times$. 
  Therefore every element of $\Q_p$ is of the form $p^k w$
  where $k \in \Z$ and $w \in \Z_p^\times$. 
  This shows $(\Z_p)_p$ surjects onto $\Q_p$,
  i.e. the canonical morphism from $(\Z_p)_p$ to $\Q_p$ is an isomorphism.
\end{proof}

\begin{rmk}[Meaning of $\Q_p$]\link{rmk:padic:rat}
  
  Continuing with the \linkto{rmk:padic}{analogy}, 
  $\Q_p$ is the field of Laurent series at $p$ 
  with $p$ as a non-essential singularity.
\end{rmk}

\begin{dfn}[$p$-adic Valuation on $\Q_p$]\link{dfn:padic:rat:val}
  
  We extend the $p$-adic valuation to $\Q_p$ by : 
  \begin{align*}
    v_p : \map{\Q_p}{\N^\infty}{}{}, 
    \frac{x}{p^n} \in (\Z_p)_p \mapsto v_p(x) - n
  \end{align*}

  From this, we extend the $p$-adic norm as well : 
  \begin{align*}
    \abs{\hspace{5pt}}_p : \map{\Q_p}{[0,\infty)\subs \R}{}{}, 
    x \mapsto \begin{cases}
      p^{-v_p(x)} &,\, x \neq 0 \\
      0 &,\, x = 0
    \end{cases}
  \end{align*}
\end{dfn}

\begin{prop}[Topological Properties of $\Q_p$]\link{prop:padic:rat:top_prop}
  
  The following are true : 
  \begin{enumerate}
    \item $(\Q_p,\abs{\hspace{5pt}}_p)$ is a normed ring 
    and hence a topological ring (field). 
    \item $\Z_p$ is homeomorphic to its canonical image in $\Q_p$,
    where it is an open subring of $\Q_p$.
    Hence, $\Q_p$ is locally compact. 
    \item $\Q_p$ is complete. 
    \item Since $\Z$ injects canonically into $\Q_p$, 
    $\Q$ injects canonically into $\Q_p$ as well. 
    Then $\Q$ is dense $\Q_p$.
  \end{enumerate}
\end{prop}
\begin{proof}~
  
  $(1)$ Same proof as for $\Z_p$.
  
  $(2)$ Since the norm of $\Q_p$ extends that of $\Z_p$, 
  $\Z_p$ is homeomorphic to its canonical image in $\Q_p$. 
  $\Z_p = B_{p}(0)$, since the image of $\abs{\hspace{5pt}}_p$ is discrete. 
  For all points $x \in \Q_p$, 
  the clopen ball of size $1$ around $x$ is homeomorphic to $\Z_p$ 
  (by translation).
  Hence every $x$ has a compact neighbourhood. 

  $(3)$ Let $a : \N \to \Q_p$ be a cauchy sequence. 
  Then there exists $N \in \N$ such that for all $n \geq N$, 
  $a_n \in B_1(a_N) = a_N + \Z_p$. 
  Since $\Z_p$ is complete and $B_1(a_N)$ is isometric to $\Z_p$,
  $a_n$ converges in $B_1(a_N)$
  and hence in $\Q_p$. 

  $(4)$ follows from elements in $\Q_p$ being 
  of \linkto{prop:padic:rat:localization}{the form $p^{-n} x$}
  where $x \in \Z_p$ and 
  \linkto{prop:padic:top_prop}{$\Z$ being dense in $\Z_p$}. 
\end{proof}

\subsection{\texorpdfstring{$p$-adic Equations}{TEXT}}

The goal of this section give 
conditions to lift approximate solutions mod $p^n$ to solutions in $\Z_p$. 
This will be done via the $p$-adic analogue of 
\linkto{prop:eqn:newton}{Newton's method}.

\begin{prop}[Inverse Limit of Finite, Non-Empty System is Non-Empty]
  \link{prop:inv_lim:nonempty}
  
  Let $D : \N\op \to \SET$ be a projective system 
  such that for all $n\in \N$, $D_n$ is finite and non-empty. 
  Then $\LIM D $ is nonempty. 
\end{prop}
\begin{proof}
  If $D$ is a surjective system, then $\LIM D $ is non-empty. 
  We will reduce to this case. 

  For $n \in \N$, 
  consider the descending sequence of subsets 
  $\set{\fall{n}{k} D_k \st n \leq k}$ in $D_n$. 
  Since $D_n$ is finite, 
  there exists an $N$ such that for all $k \geq N$, 
  $\fall{n}{k} D_k = \fall{n}{N} D_N$.
  For $n \in \N$, 
  let $N(n)$ be the minimal natural with respect to this property. 
  Let $E_n := \fall{n}{N(n)} D_{N(n)}$.
  Since $D_{N(n)} \neq \nothing$, $E_n \neq \nothing$. 
  For $n \in \N$, 
  let $M = \max(N(n),N(n+1))$. 
  Then $E_n = \fall{n}{M} D_M = \fall{n}{n+1}\fall{n+1}{M} D_M 
  = \fall{n}{n+1} E_{n+1}$. 
  Thus $E : \N\op \to \SET$ is a non-empty, surjective system that
  injects into $D$. 
  Therefore $\nothing \neq \LIM E \to \LIM D$.
\end{proof}

\begin{notation}
  Let $n \in \N, 0 < m$. 
  Then there is a canonical morphism of $\Z_p$ algebras from 
  $\Z_p[X_1,\dots,X_m]$ to $\Z/p^n\Z[X_1,\dots,X_m]$.
  For $I \subs \Z_p[X_1,\dots,X_m]$,
  let $I_n$ denote the image of $I$.
  For a single polynomial $f \in \Z_p[X_1,\dots,X_m]$,
  let $f_n$ denote its image. 
  More explicitly, for $f = \dsum{t \in \N^m}{} a_t \underline{X}^t$,
  \[
    f_n := \dsum{t \in \N^m}{} \ep_n(a_t) \underline{X}^t
  \]
\end{notation}

\begin{dfn}[Vanishing]\link{dfn:vanish}
  
  Let $A$ be a ring, $m \in \N$, $I \subs A[X_1,\dots,X_m]$. 
  Then $\V_A(I) \subs A^m$ is defined as the tuples $x$ such that 
  for all $f \in I$, $f(x) = 0$. 
  When the ring in question is clear, 
  we abbreviate to $\V(I)$. 
\end{dfn}

\begin{prop}[$p$-adic Affine Variety is Inverse Limit]
  \link{prop:eqn:inv_lim}

  Let $0 < m$, $I \subs \Z_p[X_1,\dots,X_m]$, $I_n$ the image of $I$
  in $\Z/p^n\Z[X_1,\dots,X_m]$ for $n \in \N$. 
  Then $\V(I) \iso \LIM \V(I_\star)$ as sets. 
  In particular, 
  the variety defined by $I$ is non-empty if and only if 
  for all $n\in \N$, its projection mod $p^n$ is non-empty. 
\end{prop}
\begin{proof}
  
  %We first show that $\Z_p^m$ has the universal property of 
  %$\LIM (\Z/p^\star\Z)^m$. 
  %Let $X$ be an arbitrary set. 
  %We have the following chain of set-theoretic isomorphisms : 
  %\begin{align*}
  %  \SET(X,\Z_p^m) \iso \brkt{\SET(X,\Z_p)}^m 
  %  \iso \brkt{\SET^{\N\op}(\underline{X},\Z/p^\star\Z)}^m 
  %  \iso \SET^{\N\op}(\underline{X},\brkt{\Z/p^\star\Z}^m)
  %\end{align*}
  % (Breaking this fall : 
  % An element $x$ of $\Z_p^m$ is the same as 
  % $m$ elements $x_i$ of $\Z_p$,
  % which is the same as an $m$-tuple of sequences 
  % $((\ep_k(x_1))_{k \in \N},\dots,\ep_k(x_m))_{k \in \N})$ in 
  % the projective system $\Z/p^\star\Z$, 
  % which is the same as a sequence 
  % $((\ep_k(x_1),\dots,\ep_k(x_m)))_{k \in \N}$ in 
  % the projective system $(\Z/p^\star\Z)^m$, 
  % which is the same as an element in the inverse limit of $(\Z/p^\star\Z)^m$.
  % )
  %Thus $\Z_p^m \iso \LIM (\Z/p^\star\Z)^m$ in a unique way that 
  %commutes with their projections to $(\Z/p^n\Z)^m$. 
  First note that 
  since limits commute with limits, 
  $\Z_p^m \iso \LIM \brkt{\zmo{\star}}^m$.

  For $x \in \Z_p^m$ and $f \in \Z_p[X_1,\dots,X_m]$,
  $f(x) = 0$ if and only if for all $n \in \N$, $\ep_n\circ f(x) = 0$. 
  For $n \in \N$, 
  let $\ep_n^m : \Z_p^m \to \zmo{n}^m$ denote the natural projection.
  Then \[
    \ep_n\circ f(x) = \ep_n\brkt{\dsum{t\in\N^m}{}a_tx^t}
    = \dsum{t\in\N^m}{}\ep_n(a)\ep_n^m(x)^t
    = f_n\circ\ep_n^m(x)
  \]
  Therefore $f(x) = 0$ if and only if 
  for all $n\in\N$, $f_n\circ\ep_n^m(x) = 0$.
  This shows that $\V(I) \iso \LIM \V(I_\star)$
  under the isomorphism $\Z_p^m \iso \LIM (\Z/p^\star\Z)^m$. 

  The `in particular' follows from 
  \linkto{prop:inv_lim:nonempty}{inverse limit of finite, nonempty is nonempty}.
\end{proof}

\begin{dfn}[Primitive Solutions]\link{dfn:prim}
  
  Let $m, n \in \N^+$. 
  Let $\ep_1^m : \Z_p^m \to \zmo{}^m$ and 
  $(\fall{1}{n})^m : \zmo{n}^m \to \zmo{}^m$ be the natural projections. 
  For $x \in \Z_p^m$, $x$ is called \emph{primitive} when $\ep_1^m(x) \neq 0$,
  i.e. when it is not divisible by $p$. 
  Similarly, for $x \in (\Z/p^n\Z)^m$, 
  $x$ is called primitive when $(\fall{1}{n})^m x \neq 0$. 
\end{dfn}

\begin{dfn}[Homogeneous Polynomials]\link{dfn:homogen}
  
  Let $1 \leq m$, $A$ be a commutative ring,
  $f \in A[X_1,\dots,X_m]$. 
  Then $f$ is called \emph{homogeneous} when 
  for all $\la \in A$, $f(\la X) = \la^{\deg f} f(X)$.
  Equivalently, all monomials in $f$ with non-zero coefficients 
  have the same degree. 
\end{dfn}

\begin{prop}[$\Q_p$, $\Z_p$ Points of Projective Varieties]
  \link{prop:eqn:rat_pts}

  Let $1\leq m$, $I \subs \Z_p[X_1,\dots,X_m]$,
  for all $f \in I$, $f$ homogeneous. 
  Then the following are equivalent : 
  \begin{enumerate}
    \item There exists $x \in \V_{\Q_p}(I)$ such that $x \neq 0$.
    \item There exists $x \in \V_{\Z_p}(I)$ such that $x$ is primitive.
    \item For all $n \geq 1$, 
    there exists $x_n \in \V_{\Z/p^n\Z}(I_n)$ such that $x_n$ primitive.
    \footnote{
      Serre only requires $n > 1$. 
      This is indeed equivalent since have a primitive zero for any $n > 1$
      automatically gives you a primitive zero for $n = 1$ via $\fall{1}{n}$. 
      We cannot let $n = 0$ though, since there are no primitive elements in 
      ${\Z/\Z}^m = 0^m$. 
    }
  \end{enumerate}
\end{prop}
\begin{proof}~
  
  $(1\iff 2)$ The reverse implication is clear. 
  For forwards, let $x = (x_i)_{i = 1}^m \in \V_{\Q_p}(I)$, $x \neq 0$. 
  Let $h := \inf\set{v_p(x_i) \st i = 1, \dots, m}$.
  Since $x \neq 0$, $h < \infty$.
  Let $y := p^{-h}x$. 
  Then by definition of $h$, $y \in \Z_p^m$ and there exists one component 
  that is not-divisible by $p$, i.e. $y$ is primitive. 
  Then $f(y) = p^{-h\deg f}f(x) = 0$ by homogeneity of $f$. 
  Thus $y$ is as desired.
  
  $(2 \iff 3)$ It suffices to show that 
  the sets of primitive elements in $\V_{\Z/p^n\Z}\brkt{I_n}$ forms a 
  projective subsystem of $\V_{\Z/p^\star\Z}\brkt{I_\star}$ and that 
  the inverse limit is isomorphic to the primitive elements in 
  $\V_{\Z_p}\brkt{I}$.
  
  Let $P : \N\op \to \SET$, 
  $n \mapsto \V_{\Z/p^n\Z}\brkt{I_n} \cap \set{x \st x \text{ primitive}}$. 
  By the definition of $\V_{\Z/p^\star\Z}\brkt{I_\star}$ being projective, 
  ${\fall{n}{n+1}}^m$ takes primitive zeros to primitive zeros.
  This induces the structure of a projective system for $P$,
  making it a subsystem of $\V_{\Z/p^\star\Z}\brkt{I_\star}$.
  Hence, $\LIM P$ injects into $\V_{\Z_p}\brkt{I}$ canonically. 
  We identify it with its image. 
  Clearly, for any $x \in \LIM P$, $\ep_1(x) \neq 0$. 
  So $\LIM P$ is a subset of primitive elements of $\V_{\Z_p}\brkt{I}$.
  Conversely, any primitive element $x$ of $\V_{\Z_p}\brkt{I}$ defines 
  a natural transformation from the singleton set $*$ as a constant functor
  to the projective system $P$,
  i.e. an element of $\LIM P$ that maps to $x$. 
  Hence $\LIM P$ is equal to 
  the set of primitives in $\V_{\Z_p}\brkt{I}$.
\end{proof}

\begin{prop}[Mean Value Theorem for Polynomials]
  \link{prop:poly:mvt}
  Let $A$ be a commutative ring, $f \in A[X]$, $a \in A$. 
  Then $f - f(a) = f'(a)(X - a)$ in $A[X]/(X-a)^2A[X]$. 
\end{prop}
\begin{proof}
  If the result is true for $g, h \in A[X]$,
  then it's true for $\la g + h$ where $\la \in A$. 
  Therefore it suffices to show the result for monomial $X^n$. 
  This follows from induction. 
\end{proof}

\begin{prop}[$p$-adic Newton's Method]\link{prop:eqn:newton}
  
  Let $f \in \Z_p[X]$, $x \in \Z_p$ such that 
  \[
    \abs{f(x)}_p < \abs{f'(x)}_p^2
  \]
  Then there exists $\bar{x} \in \Z_p$ such that 
  \begin{enumerate}
    \item $\abs{f(\bar{x})}_p \leq p^{-1}\abs{f(x)}_p$
    \item $\abs{\bar{x} - x}_p \leq \frac{\abs{f(x)}_p}{\abs{f'(x)}_p}$
    \item $\abs{f'(\bar{x})}_p = \abs{f'(x)}_p$
  \end{enumerate}
\end{prop}
\begin{proof}
  If $f(x) = 0$, then pick $\bar{x} = x$. 
  So WLOG $0 < \abs{f(x)}_p$. 
  Note that since all $p$-adic integers have 
  norm $\leq 1$, we have $\abs{f(x)}_p < \abs{f'(x)}_p$. 
  Then $1 < \abs{f'(x)}_p \abs{f(x)}_p\inv \in p\Z \subs p\Z_p$. 
  Define \[
    \bar{x} := x + \frac{\abs{f'(x)}_p}{\abs{f(x)}_p} y
  \]
  for some $y \in \Z_p$ to be determined. 
  Then by applying mean value theorem to $f$, we have \begin{align*}
    f(\bar{x}) 
    &= f(x) + f'(x)(\bar{x} - x) + a_0 (\bar{x} - x)^2 \\
    &= f(x) + f'(x) y \abs{f'(x)}_p \abs{f(x)}_p\inv
    + a \abs{f'(x)}_p^2 \abs{f(x)}_p^{-2}
  \end{align*}
  for some $a, a_0 \in \Z_p$.
  By definition of $\abs{\hspace{5pt}}_p$, 
  the \linkto{prop:padic:top_prop}{topology of $\Z_p$} and 
  \linkto{prop:padic:decomp}{unique decomposition}, 
  $f(x) = b \abs{f(x)}_p\inv$ for some $b \in \Z_p^\times$
  and $f'(x) = c \abs{f'(x)}_p\inv$ for some $c \in \Z_p^\times$.
  We thus have \[
    f(\bar{x}) = \brkt{b + y c}\abs{f(x)}_p\inv 
    + a \abs{f'(x)}_p^2 \abs{f(x)}_p^{-2}
  \]
  Choosing $y := -bc\inv$, we obtain : 
  \begin{align*}
    &\abs{f(\bar{x})}_p = \abs{a \abs{f'(x)}_p^2 \abs{f(x)}_p^{-2}}_p
    \leq \abs{f(x)}_p^2 \abs{f'(x)}_p^{-2} < \abs{f(x)}_p 
    \implies \abs{f(\bar{x})}_p \leq p\inv \abs{f(x)}_p \\
    &\abs{f'(x)}_p \abs{\bar{x} - x}_p
    = \abs{f(\bar{x}) - f(x) - a_0 (\bar{x}-x)^2}_p
    \leq \max(\abs{f(\bar{x})}_p,\abs{f(x)_p},\abs{a_0 (\bar{x}-x)^2}_p)
    = \abs{f(x)}_p 
  \end{align*}
  The implication followed from 
  $\abs{\Z_p}_p = \set{1, p\inv, p^{-2}, \dots, 0}$.
  It remains to show $\abs{f'(\bar{x})}_p = \abs{f'(x)}_p$. 
  By applying \linkto{prop:poly:mvt}{mean value theorem} to $f'$, 
  we have for some $d, e \in \Z_p$, \begin{align*}
    f'(\bar{x})
    &= f'(x) + f''(x) y \abs{f'(x)}_p\abs{f(x)}_p\inv 
    + d  \abs{f'(x)}_p^2 \abs{f(x)}_p^{-2} \\
    &= \abs{f'(x)}_p\inv (c + e \abs{f'(x)}^2\abs{f(x)}_p\inv 
    + d \abs{f'(x)}_p^3 \abs{f(x)}_p^{-2}) 
  \end{align*}
  Since $\abs{e \abs{f'(x)}^2\abs{f(x)}_p\inv}_p \leq 
  \abs{f(x)}\abs{f'(x)}_p^{-2} < 1$ and 
  $\abs{d \abs{f'(x)}_p^3 \abs{f(x)}_p^{-2}}_p \leq 
  \abs{f(x)}_p^{2} \abs{f'(x)}_p^{-4} < 1$,
  the term being multiplied by $\abs{f'(x)}_p\inv$ is still a unit, 
  and hence norm $1$. 
  It then follows from taking norms that 
  $\abs{f'(\bar{x})}_p = \abs{\abs{f'(x)}_p\inv}_p = \abs{f'(x)}_p$.
\end{proof}

\begin{prop}[Lifting Solutions / Generalized Hensel's Lemma]
  \link{prop:eqn:lift}
  
  Let $1 \leq m$, $f \in \Z_p[X_1,\dots,X_m]$, $x \in \Z_p^m$ such that 
  there exists $1 \leq j \leq m$ satisfying 
  \[
    \abs{f(x)}_p < \abs{\res{\del{f}{X_j}}{x}}_p^2
  \]
  Then there exists $y \in \Z_p^m$ such that 
  $f(y) = 0$ and 
  \[
    \max(\abs{\pi_i(y - x)}_p)_{1\leq i \leq m} 
    \leq \frac{\abs{f(x)}_p}{\abs{\res{\del{f}{X_j}}{x}}_p}
  \]
  where $\pi_i : \Z_p^m \to \Z_p$ takes the $i$-th component.
\end{prop}
\begin{proof}
  We induct on $m$. 

  Suppose $m = 1$. Define $x_0 := x$. 
  Then $\abs{f(x_0)}_p < \abs{f'(x_0)}_p^2$, 
  so by \linkto{prop:eqn:newton}{$p$-adic Newton's method}, 
  we have $x_1 \in \Z_p$ such that 
  \begin{align*}
    \abs{f(x_1)}_p \leq p^{-1}\abs{f(x_0)}_p , &&
    \abs{x_1 - x_0}_p 
    \leq \frac{\abs{f(x_0)}_p}{\abs{f'(x_0)}_p} , &&
    \abs{f'(x_1)}_p = \abs{f'(x_0)}_p 
  \end{align*}
  Then $\abs{f(x_1)}_p < \abs{f'(x_1)}_p^2$. 
  By induction, we have a sequence $x : \N \to \Z_p$ such that 
  for all $k \in \N$, 
  \begin{align*}
    \abs{f(x_{k+1})}_p \leq p^{-1}\abs{f(x_k)}_p 
    \leq p^{-(k+1)} \abs{f(x_0)}_p ,  &&
    \abs{x_{k+1} - x_k}_p 
    \leq \frac{\abs{f(x_k)}_p}{\abs{f'(x_k)}_p} , && 
    \abs{f'(x_{k+1})}_p = \abs{f'(x_k)}_p
  \end{align*}
  We see that 
  $\lim_{k\to\infty} f(x_k) = 0$. 
  Furthermore, from the \linkto{prop:padic:normed_ring}{ultrametric property} 
  of $\abs{\hspace{5pt}}_p$, 
  there exists $y \in \Z_p$ such that $\lim_{k\to\infty} x_k = y$. 
  Since $\Z_p$ is a topological ring with topology from 
  $\abs{\hspace{5pt}}_p$, 
  the fact that $\Z_p \to \Z_p, x \mapsto f(x)$ is defined by finitely many 
  additions and multiplications implies
  it is continuous and hence
  $f(y) = f(\lim_{k\to\infty} x_k) = \lim_{k\to\infty} f(x_k) = 0$.
  For $k \in \N$, again by 
  the \linkto{prop:padic:normed_ring}{ultrametric property} and 
  induction on $k$,
  we have
  \[
    \abs{x_{k+1} - x}_p 
    \leq \max\brkt{\abs{x_{k+1} - x_k}_p, \abs{x_k - x}_p}
    \leq \max\brkt{\frac{\abs{f(x_0)}_p}{p^k \abs{f'(x_0)}_p},
    \frac{\abs{f(x_0)}_p}{\abs{f'(x_0)}_p}
    }
    \leq \frac{\abs{f(x_0)}_p}{\abs{f'(x_0)}_p}
  \]
  Taking limits, we obtain \[
    \abs{y - x}_p \leq \frac{\abs{f(x_0)}_p}{\abs{f'(x_0)}_p}
  \]
  as desired. 

  For $1 < m$, we reduce to the single variable case. 
  Define $\bar{f}(X_j) := f(\pi_1(x),\dots,X_j,\dots,\pi_m(x)) \in \Z_p[X_j]$.
  By the single variable case,
  there exists $y_j \in \Z_p$ such that $\bar{f}(y_j) = 0$ and 
  \[
    \abs{y_j - \pi_j(x)}_p 
    \leq \frac{\abs{\bar{f}(\pi_j(x))}_p}{\abs{\bar{f}'(\pi_j(x))}_p}
    = \frac{\abs{f(x)}_p}{\abs{{f}'(x)}_p}
  \]
  Let $y = (\pi_1(x),\dots,y_j,\dots,\pi_m(x))\in \Z_p^m$.
  Then $f(y) = \bar{f}(y_j) = 0$ and for all $1 \leq i \leq m$,
  \[
    \abs{\pi_i(y - x)}_p \begin{cases}
      = 0 &\, i \neq j \\
      \leq \frac{\abs{f(x)}_p}{\abs{f'(x)}_p} &\, i = j
    \end{cases} 
  \]
\end{proof}

\begin{prop}[Hensel's Lemma]\link{prop:eqn:hensel}
  
  Let $1 \leq m$, $f \in \Z_p[X_1,\dots,X_m]$, $x \in \Z_p^m$, 
  $\ep_1(f(x)) = 0$, $1 \leq i \leq m$, $\ep_1(\res{\del{f}{X_i}}{x}) \neq 0$.
  Then there exists $y \in \Z_p^m$ such that 
  $f(y) = 0$ and $\ep_1^m(y - x) = 0$. 
\end{prop}
\begin{proof}
  $\ep_1(f(x)) = 0$ is equivalent to $\abs{f(x)}_p \leq p\inv$
  and $\ep_1(\res{\del{f}{X_i}}{x}) \neq 0$ is equivalent to 
  $\abs{\res{\del{f}{X_i}}{x}}_p = 1$.
  The conditions of \linkto{prop:eqn:lift}{lifting solutions} are satisfied,
  hence we have $y \in \Z_p^m$ such that for all $1 \leq i \leq m$, 
  \[
    \max(\abs{\pi_i(y - x)}_p)_{1\leq i \leq m} 
    \leq \frac{\abs{f(x)}_p}{\abs{\res{\del{f}{X_j}}{x}}_p}
  \]
  The inequality is equivalent to $\ep_1^m(y - x) = 0$. 
\end{proof}

\begin{prop}[Lifting Solutions of Quadratic Forms for $p \neq 2$]
  \link{prop:eqn:lift:quad:p_neq_2}
  
  Let $p \neq 2$, $1 \leq m$, 
  $f = \dsum{i,j = 1}{m} a_{ij}X_iX_j \in \Z_p[X_1,\dots,X_m]$ where 
  \begin{enumerate}
    \item $[a_{ij}]^\top = [a_{ij}]$
    \item $\det [a_{ij}] \in \Z_p^\times$
  \end{enumerate}
  i.e. $f$ is a non-degenerate quadratic form. 
  Let $a \in \Z_p$, $x \in \Z_p^m$ such that 
  $x$ is primitive and $\ep_1(f(x)) = \ep_1(a)$. 
  Then there exists $y \in \Z_p^m$ such that 
  $f(y) = a$ and $\ep_1^m(y - x) = 0$. 
\end{prop}
\begin{proof}
  By \linkto{prop:eqn:hensel}{Hensel's Lemma},
  it suffices to give $1 \leq i \leq m$ such that 
  $\ep_1(\res{\del{f}{X_i}}{x}) \neq 0$. 
  Taking the derivative of $f$, 
  evaluating at $x$ and reducing mod $p$ yields 
  the following linear system : 
  \begin{align*}
    \sqbrkt{\ep_1\brkt{\res{\del{f}{X_i}}{x}}}_{i = 1}^m 
    = 2 [\ep_1(a_{ij})]_{i,j = 1}^m \ep_1(x)
  \end{align*}
  Since $\det [a_{ij}] \in \Z_p^\times$, 
  $\det [\ep_1(a_{ij})]_{i,j = 1}^m \neq 0$.
  The matrix is hence invertible and since 
  $\ep_1(x) \neq 0$ by definition of \linkto{dfn:padic:prim}{primitivity},
  there exists a desired $1 \leq i \leq m$. 
\end{proof}

\begin{prop}[Lifting Solutions of Quadratic Forms for $p = 2$]
  \link{prop:eqn:lift:quad:p_eq_2}
  
  Let $p = 2$, $1 \leq m$, 
  $f = \dsum{i,j = 1}{m} a_{ij}X_iX_j \in \Z_p[X_1,\dots,X_m]$ where 
  $[a_{ij}]^\top = [a_{ij}]$, i.e. $f$ is a quadratic form. 
  Let $a \in \Z_2$, $x \in \Z_2^m$ such that 
  $x$ is primitive and $\ep_3(f(x)) = \ep_3(a)$. 
  Then \begin{enumerate}
    \item Let $1 \leq i \leq m$ where 
    $\ep_2\brkt{\res{\del{f}{X_i}}{x}} \neq 0$.
    Then there exists $y \in \Z_2^m$ such that $f(y) = a$ and 
    $\ep_3(y - x) = 0$. 
    \item The condition of $(1)$ is satisfied when 
    $\det\sqbrkt{a_{ij}}_{i,j = 1}^m \in \Z_2^\times$. 
  \end{enumerate}
\end{prop}
\begin{proof}~

  $(1)$
  $\ep_3(f(x)) = \ep_3(a)$ and 
  $\ep_2\brkt{\res{\del{f}{X_i}}{x}} \neq 0$ are respectively equivalent to
  $\abs{f(x) - a}_p \leq p^{-3}$ and $p\inv \leq \abs{\res{\del{f}{X_i}}{x}}_p$.
  Hence \[
    \abs{f(x) - a}_p < \abs{\res{\del{f}{X_i}}{x}}_p^2
  \]
  So by \linkto{prop:eqn:lift}{lifting solutions}, 
  there exists $y \in \Z_p^m$ such that 
  $f(y) = a$ and \[
    \max(\abs{\pi_i(y - x)}_p)_{1\leq i \leq m} 
    \leq \frac{\abs{f(x) - a}_p}{\abs{\res{\del{f}{X_j}}{x}}_p}
  \]
  By taking the derivative of $f$, evaluating at $x$ and 
  reducing mod $2$, we have $\ep_1\brkt{\res{\del{f}{X_i}}{x}} = 0$ 
  and hence its valuation is $1$.
  We thus obtain \[
    \max(\abs{\pi_i(y - x)}_p)_{1\leq i \leq m} 
    \leq p^{-2}
  \]
  This is equivalent to $\ep_2(y - x) = 0$.

  $(2)$ This follows from taking the derivative of $f$,
  evaluating at $x$ and reducing mod $4$, we have 
  \begin{align*}
    \sqbrkt{\ep_2\brkt{\res{\del{f}{X_i}}{x}}}_{i = 1}^m 
    = 2 [\ep_2(a_{ij})]_{i,j = 1}^m \ep_2(x)
  \end{align*}
  Since $\det [a_{ij}]_{i,j = 1}^m \in \Z_2^\times$, 
  $\det [\ep_2(a_{ij})]_{i,j = 1}^m \in \zmo[2]{2}^\times$,
  the fact that $\ep_2(x)$ is not a multiple of $2$ implies the tuple
  $[\ep_2(a_{ij})]_{i,j = 1}^m \ep_2(x)$ is not a multiple of $2$.
  The existence of $i$ such that $\ep_2\brkt{\res{\del{f}{X_i}}{x}} \neq 0$
  follows. 
\end{proof}

\subsection{Appendix : Omitted Proofs}

\begin{proof}(\linkto{prop:inv_lim:left_surj_imp_right_exact}{
  Left Surjective implies Right Exactness of Inverse Limit
} - from Atiyah)

  It is elementary to check that we have the short exact sequence of $R$-modules
  \[
    0 \to \prod A \to \prod B \to \prod C \to 0
  \]
  where $\prod$ takes the product of $R$-modules. 
  For $n \in \N$, let $\pi_n : \prod A \to A_n$ be the natural projection. 
  Note that we have a canonical map \begin{cd}
    \LIM A \arrow[r,"\prod \ep"] & \prod A
  \end{cd}
  induced from the natural maps $\ep_n : \LIM A \to A_n$ for $n \in \N$. 
  Define $d^A : \prod A \to \prod A$ via 
  $d^A_n : \prod A \to A_n := \fall{n}{n+1} \pi_{n+1} - \pi_n$
  and the universal property of $\prod A$. 
  Define $d^B, d^C$ similarly for $B, C$ and we have 
  the following commutative diagram of $R$-modules with exact rows : 
  \begin{cd}[sep = small]
    0 \arrow[r] & 
    \prod A \arrow[r] \arrow[d,"d^A"]& 
    \prod B \arrow[r] \arrow[d,"d^B"] & 
    \prod C \arrow[r] \arrow[d,"d^C"]& 
    0 \\
    0 \arrow[r] & 
    \prod A \arrow[r] & 
    \prod B \arrow[r] & 
    \prod C \arrow[r] & 
    0
  \end{cd}
  Applying the snake lemma, we obtain exact sequence of $R$-modules : 
  \begin{cd}
    0           \arrow[r] & 
    \ker d^A    \arrow[r] & 
    \ker d^B    \arrow[r] & 
    \ker d^C    \arrow[r] & 
    \coker d^A  \arrow[r] & 
    \coker d^B  \arrow[r] & 
    \coker d^C  \arrow[r] & 
    0
  \end{cd}
  It is straight forward to check that 
  $\LIM A, \LIM B, \LIM C$ are respectively the kernels of $d^A, d^B, d^C$
  and that $A$ surjective implies the zero module $0$ is the cokernel of $d^A$.
  The result follows. 
\end{proof}

\begin{proof}~(\linkto{prop:padic:truncation}{Truncation} - from Serre)

  \textit{(Exactness at left)}
  It suffices to show that multiplying by $p$ is an injection,
  i.e. you can cancel by $p$. 
  Let $x \in \Z_p$ such that $p x = 0$. 
  Then for $k \in \N$, $0 = \ep_{k+1}(p x) = p \ep_{k+1}(x)$ implies 
  the existence of a $x_{k+1} \in \Z$ such that 
  $x_{k+1} = \ep_{k+1}(x)$ in $\Z/p^{k+1}\Z$ and 
  $p^{k+1} \mid p x_{k+1}$. 
  Then $p^k \mid x_{k+1}$, so 
  $\ep_k(x) = \fall{k}{k+1} \ep_{k+1}(x) = \fall{k}{k+1} x_{k+1} = 0$. 
  Therefore $\ep_k(x) = 0$ for all $k \in \N$, i.e. $x = 0$. 

  \textit{(Exactness at right)} $\ep_n$ is surjective. 

  \textit{(Exactness in middle)} Clearly, $p^n\Z_p \subs \ker \ep_n$. 
  Let $x \in \ker \ep_n$.
  In the following, for $k \in \N$, 
  let $\pi_k : \map{\Z}{\Z/p^k\Z}{}{}$ be the natural projection. 
  For $k \in \N$, let $x_k$ be the unique integer in $\set{0,\dots,p^k-1}$ 
  such that $\pi_k(x_k) = \ep_k(x)$ in $\Z/p^k\Z$. 
  Then $\ep_n(x) = 0$ implies for all $k \in \N$, 
  \[ \pi_n (x_{n+k}) = \fall{n}{n+k} \pi_{n+k}(x_{n+k}) 
  = \fall{n}{n+k} \ep_{n+k}(x) = \ep_n(x) = 0 \] 
  that is to say $p^n \mid x_{n+k}$. 
  Since $0 \leq x_{n+k} < p^{n+k}$, 
  there exists a unique $0 \leq y_k < p^k$ such that 
  $x_{n+k} = p^n y_k$ in $\Z$. 
  Let $y \in \dprod{n \in \N}{} \Z/p^n\Z$ such that for all $k \in\N$, 
  $\ep_k(y) = \pi_k(y_k)$. 
  Then for $k \in \N$, 
  $\pi_{n+k}(x_{n+k+1}) = \fall{n+k}{n+k+1}\ep_{n+k+1}(x) 
  = \ep_{n+k}(x) = \pi_{n+k}(x_{n+k})$
  implies $p^{n+k} \mid x_{n+k+1} - x_{n+k} = p^n(y^{k+1} - y^k)$,
  and therefore $p^k \mid y^{k+1} - y^k$.
  Hence $y \in \Z_p$. 
  Then for $k \in \N$, 
  \[
    \ep_k(p^n y) = \pi_k(p^n y_k) = \fall{k}{n+k}\pi_{n+k}(p^n y_k)
    = \fall{k}{n+k} \pi_{n+k}(x_{n+k}) = \fall{k}{n+k} \ep_{n+k}(x)
    = \ep_k(x)
  \]
  Therefore, $x = p^n y \in p^n \Z_p$.
\end{proof}

\begin{proof}~(\linkto{prop:padic:local}{$\Z_p$ Local Ring} - from Serre)

  We first prove that for $n \geq 1$, 
  $\zmo{n}$ is a local ring with maximal ideal $p\zmo{n}$.
  Let $n \geq 1$. 
  It suffices to show that $\Z/p^n\Z \minus p\Z/p^n\Z \subs \Z/p^n\Z^\times$. 
  Let $x \in \Z/p^n\Z$ be not divisible by $p$. 
  Then there exists $y \in \Z/p^n\Z$ such that $\fall{1}{n} (x y) = 1$.
  Let $0 \leq x_n, y_n < p^n$ be representatives of $x, y$ in $\Z$. 
  Then there exists $z_n \in \Z$ such that $x_n y_n = 1 - p z_n$. 
  Let $\pi_n : \map{\Z}{\Z/p^n\Z}{}{}$ be the natural projection and 
  $z := \pi_n(z_n)$. 
  Then we have \[
    x y (1 + p z + \cdots + (pz)^{n-1})
    = \pi_n((1 - p z_n)(1 + p z + \cdots + (pz)^{n-1}))
    = \pi_n(1 - (p z_n)^n)
    = 1 
  \]
  Thus $x$ is a unit. 

  To show $\Z_p$ is a local ring with maximal ideal $p\Z_p$, 
  it again suffices that $\Z_p\minus p\Z_p \subs \Z_p^\times$. 
  Let $x \in \Z_p\minus p\Z_p$. 
  Then for all $n \geq 1$, $0 \neq \ep_1(x) = \fall{1}{n}\ep_n(x)$. 
  Since $\fall{1}{n} : (\Z/p^n\Z) / (p\Z/p^n\Z) \iso \Z/p\Z$ as rings, 
  $\ep_n(x) \in \Z/p^n\Z^\times$ by the above. 
  Let $y_n = \ep_n(x)\inv$. 
  Then uniqueness of inverses implies $\fall{n}{n+1} y_{n+1} = y_n$,
  i.e. there exists a unique $y \in \Z_p$ such that 
  for all $n$, $\ep_n(y) = y_n$.
  Then $x y = 1$, i.e. $x \in \Z_p^\times$. 
\end{proof}